\documentclass[liststotoc,a4paper, 12pt]{scrartcl}
\usepackage[utf8]{inputenc}
\usepackage{ngerman}
\usepackage{setspace}
\usepackage{geometry}
\usepackage{graphicx}
\usepackage{cite}
\usepackage{listings}
\usepackage{color}


\geometry{a4paper, top=25mm, left=25mm, right=25mm, bottom=25mm}

\title{Bachelorarbeit}
\author{Kevin Seidel \\ Studiengang Informatik \\ Matrikelnummer: 943147}

\begin{document}
\begin{titlepage}
\begin{center}
	
\includegraphics[scale=0.1]{pictures/uos_logo.png}\\
\vspace*{1.5cm}
\begin{Large}
\textbf{Universität Osnabrück}
\end{Large}

\noindent\hrulefill
\\[0.25cm]
Fachbereich Informatik \\[3.5cm]
\begin{large}\textsc{Bachelorarbeit}\end{large} \\[2cm]
\begin{huge}\textbf{Indoor Positionierung mittels Bluetooth Low Energy} \end{huge} \\[1.5cm]
Erstellt am 28.01.2014
\\[3.5cm]
\textbf{Vorgelegt von:} \\
Kevin Seidel \\
Falkenstraße 43 \\
49124 Georgsmarienhütte \\
keseidel@uni-osnabrueck.de
\\[1cm]
\textbf{Geprüft von:} \\
Prof. Dr. Oliver Vornberger \\
Prof. Dr. Elke Pulvermüller

\end{center}
\end{titlepage}


\newpage

\pagenumbering{Roman}
\setcounter{page}{1}

\setcounter{secnumdepth}{-2}
\section*{Kurzfassung}%
\addcontentsline{toc}{section}{Kurzfassung}%

\newpage

\setuptoc{toc}{totoc}


\tableofcontents


\newpage

\listoffigures
\newpage

\listoftables
\newpage

\lstlistoflistings
\newpage

\pagenumbering{arabic}
\setcounter{page}{1}
\setcounter{secnumdepth}{2}


\section{Danksagung}

\section{Einleitung}
\subsection{Motivation}
\subsection{Ziele der Bachelorarbeit}

\section{Technologien}

\subsection{Bluetooth 4.0}
\subsubsection{Bluetooth Low Energy}
\subsubsection{iBeacons}

\subsection{iOS und Xcode}

\subsection{CoreLocation-Framework}
\subsubsection{iBeacons-API}
\subsubsection{Weitere APIs}

\subsection{CoreData-Framework}


\section{Werkzeuge}
\subsection{Xcode}
\subsection{Versionsverwaltung mit Git}
\subsection{iPhone}

\section{Daten und Messungen}
\subsection{Mobile iBeacons}
\subsection{Stationäre iBeacons}
\subsection{Außenmessugen}
\subsection{Innenraummessungen}

\section{Umsetzung und Implementation}

\subsection{Ansatz zur Positionbestimmung}
\subsubsection{Trilateration}
\subsubsection{Fingerprinting}

\section{Fingerprinting}
\subsection{Positionbestimmung}
\subsubsection{Nearest-Neighbor-Verfahren}
\subsubsection{Probabilistisch-Verfahren}

\section{Fazit und Ausblick}

\section{Literatur}



\end{document}