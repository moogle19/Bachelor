\documentclass{AGTI} 

%% %%%%%%%%%%%%%%%%%%%%%%%%%%%%%%%%%%%%%%%%%%%%%%%%%%%%%%%%%%%%%%%%%%%%%%%%%%%%%%%%%
%% Hier bitte die Eckdaten der Arbeit eingeben (Arbeits-Typ, Titel, Name, Gutachter)
%% %%%%%%%%%%%%%%%%%%%%%%%%%%%%%%%%%%%%%%%%%%%%%%%%%%%%%%%%%%%%%%%%%%%%%%%%%%%%%%%%%

% Typ der Arbeit (z.B. Diplomarbeit, Projektarbeit, Seminararbeit,..):
%\subject{Dokumentation}
\typ{Dokumentation}

% Titel der Arbeit:
\titel{Arbeiten mit der \LaTeX-Vorlage f�r Abschlussarbeiten}

% Autor der Arbeit:
\autor{Kalle Kleinl�tzum, \\Nils Rosemann}


% Datum der Abgage (wird automatisch auf aktuelles Datum gesetzt)
\datum{\today}

% Die beiden Gutachter der Arbeit (f�r Bachelor, Master, Diplom)
\firstSupervisor{Prof. Werner Brockmann}
\secondSupervisor{Prof. Kalle Rosemann}
\erstgutachter{Prof. Werner Brockmann}
\zweitgutachter{Prof. Mr. X }

% Pfad zu den Bildern
\graphicspath{
 {./bilder/}
 {/noch/ein/verzeichnis}
}

% Eigene Makros
\newcommand{\latexklasse}{\texttt{AGTI.cls}}

  
\begin{document}

% F�r eine sch�nere Ausgabe f�r zu Hause oder f�r den Betreuer besteht
% hier die M�glichkeit, der Arbeit mit einem zus�tzlichen schmucken Deckblatt 
% (mit Titelbild) eine pers�nliche Note zu verleihen: 
% Dazu dem Befehl \extraTitelblatt einen \includegraphics-Befehl �bergeben
%\extraTitelblatt{\fbox{\includegraphics[width=.52\linewidth]{titelbild}}}
%\cleardoublepage

% Dieser Befehl erzeugt das offizielle Titelblatt
\standardTitelblatt
\cleardoublepage

% Dieser "normale" Titel-Erzeugungsbefehl wird nicht mehr ben�tigt
%\maketitle
%\newpage 
%\cleardoublepage

% Durch den folgenden Befehl wird die Erkl�rung eingebunden,
% dass man die Arbeit seri�s erstellt hat.

% \erklaerung%%{\today}
% \cleardoublepage

% Hier die Zusammenfassung auf Deutsch und auf Englisch
\thispagestyle{empty}
\section*{Zusammenfassung}
Bei allen Abschlussarbeiten wird zu Beginn der Arbeit eine kurze Zusammenfassung in deutscher und englischer Sprache verlangt. Diese d�rfen zusammen maximal eine Seite einnehmen.
\vfill
\section*{Abstract}
In the preface of all thesis, a short abstract of the work is required both in german and in english language. The length for both is restricted to one page maximum.
\cleardoublepage

% Hier kommt bei umfangreichen Arbeiten (Master- oder Diplomarbeit) ein Vorwort
%   \chapter*{Vorwort}
%   Da bei Master-- und Diplomarbeiten im Allgemeinen ein Vorwort erwartet
%   wird, soll beispielhaft auch bei dieser Dokumentation ein solches
%   erscheinen.  Bei Arbeiten geringeren Umfanges ist ein Vorwort
%   nicht unbedingt angebracht.
%   
%   Zumeist finden sich in Vorworten irgendwelche Danksagungen. Wem man
%   dankt, sei jedem selber �berlassen, wir m�chten an dieser Stelle 
%   Roland Bless, Tobias Luksch, Kai Lingemann und Andreas N�chter
%   danken, auf deren Vorarbeiten dieses Dokument sowie die zugeh�rige 
%   \LaTeX-Klasse basieren.
% \cleardoublepage

\pagenumbering{roman}
\tableofcontents

\cleardoublepage

%%%%%%%%%%%%%%%%%%%%%%%%%%%%%%%%%%%%%%%%%%%%%%%%%%%%%%%%%%%%
%% Beginn der eigentlichen Arbeit
%%
%% Bei l�ngeren Arbeiten ist es sinnvoll, Kapitel in 
%% einzelne Dateien auszulagern, die dann in der Hauptdatei
%% mittels \include{} eingebunden werden
%%%%%%%%%%%%%%%%%%%%%%%%%%%%%%%%%%%%%%%%%%%%%%%%%%%%%%%%%%%%

% Seitennummer auf 1 setzen

\pagenumbering{arabic}

\include{kapitel/einleitung}

\include{kapitel/latexklasse}

\include{kapitel/floats}

\include{kapitel/inhalt}

% \appendix
%   \chapter{Quellcode}
%   \section{AGTI.cls}
%   \lstset{language=[LaTeX]TeX}
%   \lstinputlisting[breaklines]{AGTI.cls}
% %\lstset{language=LaTeX}


% Dieser Stil erzeugt Verweise mit Autorenname und Jahr
%\bibliographystyle{gerapali}
\bibliographystyle{wmaainf}
%\bibliographystyle{AGTI}
\addcontentsline{toc}{chapter}{Literaturverzeichnis}

% Hier die Datei (ohne .bib) angeben, in der die referenzierten
% Paper stehen
\bibliography{literatur}
\clearpage{\cleardoublepage}

\end{document}
%%%%%%%%%%%%%%%%%%%%%%%%%%%%%%%%%%%%%%%%%%%%%%%%%%%%%%%%%%%%
%% Ende des Dokuments
%%%%%%%%%%%%%%%%%%%%%%%%%%%%%%%%%%%%%%%%%%%%%%%%%%%%%%%%%%%%
