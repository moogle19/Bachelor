%%%%%%%%%%%%%%%%%%%%%%%%%%%%%%%%%%%%%%%%%%%%%%%%%%%%%%%%%%%%
%% latexklasse.tex
%%
%% Kapitel: Die Latex-Klasse agrosy.cls
%% Hauptdokument: howto.tex
%% Autor: Tobias Luksch
%% Datum: Juli 2003
%%%%%%%%%%%%%%%%%%%%%%%%%%%%%%%%%%%%%%%%%%%%%%%%%%%%%%%%%%%%
\chapter{Aufbau der \LaTeX -Vorlage}
\label{chap:aufbau}
Dieses Kapitel behandelt die zur Verf�gung gestellte \LaTeX -Klasse
\latexklasse\ und die aus ihrer Anwendung resultierende Struktur des
Hauptdokumentes. Die wichtigsten Makros werden erl�utert und ein paar
Hinweise zum systematischen Aufbau der Arbeit gegeben.

%%%%%%%%%%%%%%%%%%%%%%%%%%%%%%%%%%%%%%%%%%%%%%%%%%%%%%%%%%%%
\section{Grundfunktionen des Hauptdokumentes}
\label{sec:aufbau:hauptdoku}
%%%%%%%%%%%%%%%%%%%%%%%%%%%%%%%%%%%%%%%%%%%%%%%%%%%%%%%%%%%%
\index{Hauptdokument}\index{\latexklasse} Das Hauptdokument (englisch
\emph{master file}), h�lt sich an die �bliche Form eines \LaTeX
-Dokumentes. Grunds�tzlich kann das Hauptdokument dieser Dokumentation
(\texttt{howto.tex}) als kommentiertes Beispiel herangezogen werden.
Als Klasse ist im Kopf des Dokumentes (\verb|\documentclass{}|)
\latexklasse\ anzugeben.
Es folgen die �blichen Angaben zu Titel, Autor und Typ der Arbeit, im
einzelnen \verb|\titel|, \verb|\autor|, \verb|\typ| (Bachelorarbeit,
Diplomarbeit,\dots) und Felder f�r die Gutachter, die bei Abschlussarbeiten
angegeben werden sollten. Dann bietet sich noch die M�glichkeit,
eigene Makros zu definieren und \index{Bilder!Verzeichnisse}
Verzeichnisse anzugeben, in denen nach Bildern gesucht werden soll
(\verb|\graphicspath|).
Anschlie�end beginnt der sichtbare Inhalt innerhalb der
\verb|document|-Umbegung. Hier k�nnen bestimmte Makros verwendet werden:
\begin{description}
\item[$\backslash$maketitle] generiert die Titelseite mit Titel und
  Autor der Arbeit
\item[$\backslash$erklaerung] erstellt eine Seite, die aussagt, dass
  der Autor die Arbeit selbst�ndig verfasst und alle benutzten Quellen
  angegeben hat. Diese Seite ist bei offiziellen Exemplaren vom Autoren zu 
  unterschreiben.
% \item[$\backslash$prefacesection] bietet die M�glichkeit, Zusammenfassung
%   und Abstract einzuf�gen. Auch Danksagungen lassen sich damit erstellen.
%   Als Beispiel sei auf die Quellen dieses Textes
%   verwiesen.
\item[$\backslash$tableofcontents] f�gt das Inhaltsverzeichnis ein.
\item[$\backslash$bibliography] erzeugt das Literaturverzeichnis. Als
  Parameter wird die zu verwendende BibTex-Datei (ohne \verb|.bib|)
  �bergeben. Mehr zu Literaturhinweisen in
  Kapitel~\ref{sec:floats:bib}.
\end{description}
%
Bei l�ngeren Arbeiten kann es sinnvoll sein, auch einen Index zu erstellen.
Dazu sei auf die einschl�gige \LaTeX-Literatur verwiesen.
Tabelle~\ref{tab:dateien} gibt eine �bersicht der zur \LaTeX -Vorlage geh�renden Dateien.
%
\begin{table}
  \centering
  \caption{Die bereitgestellten Dateien mit einer kurzen Beschreibung.} \vskip 2mm
  \begin{tabular}[c]{lp{10cm}}
    \hline
    \bf Dateiname & \bf Beschreibung \\
    \hline 
    \latexklasse & Die Klassendefinition beruhend auf scrreprt.cls \\
    \texttt{howto.tex} & Die Hauptdatei dieser Dokumentation \\
    \texttt{kapitel/} & Das Verzeichnis, in dem die Kapitel (in \texttt{tex}-Files) liegen\\
    \texttt{literatur.bib} & Die Literaturliste zu diesem Dokument im BibTeX-Format \\
    \texttt{beispiele.bib} & Kommentierte Beispieldatei zur BibTex-Datenbank \\
    \texttt{algorithmen/} & Verzeichnis mit Style-Dateien und Dokumentation zum Einbau von Algorithmen \\
%     \texttt{algorithms.ps, .tex} & Dokumentation zu den Algorithmus-Styles \\
%    \texttt{makefile} & Makefile zur automatischen Generation des Dokuments \\
    \hline
  \end{tabular}
%  \caption{Die bereitgestellten Dateien mit einer kurzen Beschreibung.}
  \label{tab:dateien}
\end{table}


%%%%%%%%%%%%%%%%%%%%%%%%%%%%%%%%%%%%%%%%%%%%%%%%%%%%%%%%%%%%
\section{Hilfreiche Tipps}
\label{sec:aufbau:tipps}
%%%%%%%%%%%%%%%%%%%%%%%%%%%%%%%%%%%%%%%%%%%%%%%%%%%%%%%%%%%%
In diesem Abschnitt werden einige Tipps zur Strukturierung und Handhabung
der Arbeit gegeben.

%%%%%%%%%%%%%%%%%%%%%%%%%%%%%%%%%%%%%%%%%%%%%%%%%%%%%%%%%%%%
\subsection{Kapitel in externen Dateien}
\label{sec:aufbau:tipps:kapitel}
%%%%%%%%%%%%%%%%%%%%%%%%%%%%%%%%%%%%%%%%%%%%%%%%%%%%%%%%%%%%
\index{Kapitel}
Bei l�ngeren Dokumenten ist es grunds�tzlich ratsam, den Text
auf mehrere Dateien zu verteilen, zum Beispiel f�r
jedes Kapitel eine eigene Datei anzulegen. Diese k�nnen dann mittels
des \verb|\include|-Befehls in der Hauptdatei eingef�gt werden. Auch
diese Dokumentation ist derart gegliedert, die einzelnen Kapitel befinden 
sich im Verzeichnis \texttt{kapitel}.

%%%%%%%%%%%%%%%%%%%%%%%%%%%%%%%%%%%%%%%%%%%%%%%%%%%%%%%%%%%%
\subsection{Label}
\label{sec:aufbau:tipps:label}
%%%%%%%%%%%%%%%%%%%%%%%%%%%%%%%%%%%%%%%%%%%%%%%%%%%%%%%%%%%%
\index{Label}
Ein Hilfsmittel, von dem man unbedingt Gebrauch machen sollte, ist die Verwendung von {\it Labeln}. Diese funktionieren so, dass jeder Abschnitt und jedes Bild mit einem eindeutigen Label versehen wird, welches zum Verweis auf den betreffenden Abschnitt bzw. das Bild verwendet wird. Dank dieser Label bleibt die Referenzierung von Dokumenteninhalten auch bei Umstellung von Abschnitten und Kapiteln immer konsistent. So kann von diesem Dokument z.B. immer guten Gewissens behauptet werden, dass Abbildung~\ref{fig:bisam} auf Seite~\pageref{fig:bisam} steht und Abschnitt~\ref{sec:aufbau:tipps:label} auf Seite~\pageref{sec:aufbau:tipps:label} beginnt.\\
Bei der Wahl der Label bietet es sich an, eine
einheitliche Benennung zu w�hlen. Diese Dokumentation benennt Label
nach dem Schema \verb|<type>:<beschreibung>|, wobei \verb|<type>| zum
Beispiel \verb|chap|, \verb|sec|, \verb|fig|, \verb|eq| oder etwas
entsprechendes sein kann. Das mag zwar alles etwas mehr Tipperei sein,
verbessert jedoch die Lesbarkeit des Dokumentes erheblich und spart 
zus�tzlich viel Arbeit.

%%%%%%%%%%%%%%%%%%%%%%%%%%%%%%%%%%%%%%%%%%%%%%%%%%%%%%%%%%%%
\subsection{Kommentare}
\label{sec:aufbau:tipps:kommentare}
%%%%%%%%%%%%%%%%%%%%%%%%%%%%%%%%%%%%%%%%%%%%%%%%%%%%%%%%%%%%
\index{Kommentare}
In \LaTeX\ sind Kommentare erlaubt (beginnend mit \verb|%|), also
sollte man davon Gebrauch machen. Hilfreich sind deutliche Einschnitte
zwischen Unterkapiteln oder anderen relevanten Abschnitten, so dass
man sich schneller in seinen Quellen zurechtfindet.

Verwendet man unter Linux (X-)Emacs zum Schreiben der Arbeit, 
l�sst sich in den Unterdateien durch einen Block der Art
\begin{verbatim}
   %%% Local Variables: 
   %%% mode: latex
   %%% TeX-master: "howto"
   %%% End: 
\end{verbatim}
am Ende der Datei auf die Hauptdatei verweisen. Das Programm weiss
dadurch, welche Datei zu kompilieren bzw. anzuzeigen ist. Verwendet
man unter Windows das Programm TeXnicCenter l�sst sich eine �hnliche
Funktionalit�t mittels Projekten erreichen.

%%%%%%%%%%%%%%%%%%%%%%%%%%%%%%%%%%%%%%%%%%%%%%%%%%%%%%%%%%%%
\subsection{Kompilieren des Dokumentes}
\label{sec:aufbau:tipps:kompilieren}
%%%%%%%%%%%%%%%%%%%%%%%%%%%%%%%%%%%%%%%%%%%%%%%%%%%%%%%%%%%%
\index{Kompilieren}
Kurz ein paar Worte zum Kompilieren der \LaTeX -Quellen. Unter Windows
und dem TeXnicCenter sollte das gr��tenteils automatisch passieren.
Unter Linux kann es sein, dass man mehrfach kompilieren muss, um die
Referenzen richtig zu kriegen. Ein komplettes Kompilieren sieht zum
Beispiel so aus:
\begin{verbatim}
   > latex howTo.tex
   > makeindex howTo.idx
   > bibtex howTo.aux
   > latex howTo.tex
   > latex howTo.tex
   >
\end{verbatim} 
Diese Schritte lassen sich aber auch durch ein \emph{makefile}, ein
\emph{SConscript} oder Shellskripte automatisieren.

%%% Local Variables: 
%%% mode: latex
%%% TeX-master: "howto"
%%% End: 
