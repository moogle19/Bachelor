%%%%%%%%%%%%%%%%%%%%%%%%%%%%%%%%%%%%%%%%%%%%%%%%%%%%%%%%%%%%
%% einleitung.tex
%%
%% Kapitel: Einleitung
%% Hauptdokument: howto.tex
%% Autor: Tobias Luksch
%% Datum: Juli 2003
%%%%%%%%%%%%%%%%%%%%%%%%%%%%%%%%%%%%%%%%%%%%%%%%%%%%%%%%%%%%
\chapter{Einleitung}
\label{chap:einl}

%%%%%%%%%%%%%%%%%%%%%%%%%%%%%%%%%%%%%%%%%%%%%%%%%%%%%%%%%%%%
\section{Motivation}
\label{sec:einl:motiv}
%%%%%%%%%%%%%%%%%%%%%%%%%%%%%%%%%%%%%%%%%%%%%%%%%%%%%%%%%%%%
Die Arbeitsgruppe Technische Informatik an der
Universit�t Osnabr�ck bietet eine Vielzahl von Angeboten an
studentischen Arbeiten an, seien es Abschlussarbeiten,
Seminare oder Praktika. F�r all diese Arbeiten ist ein Dokument zu
erstellen, das die Anspr�che an eine wissenschaftliche Arbeit erf�llen
sollte. Um es den Studierenden einfacher zu machen, diesen Anspr�chen
gerecht zu werden, bietet die Arbeitsgruppe \LaTeX -Vorlagen an,
die ein professionelles und einheitliches Layout erm�glichen.
% Diese Dokumentation 
% beschreibt die Formatvorlage f�r \emph{Abschlussarbeiten}.

%%%%%%%%%%%%%%%%%%%%%%%%%%%%%%%%%%%%%%%%%%%%%%%%%%%%%%%%%%%%
\section{Ziel der Arbeit}
\label{sec:einl:ziel}
%%%%%%%%%%%%%%%%%%%%%%%%%%%%%%%%%%%%%%%%%%%%%%%%%%%%%%%%%%%%
Das Ziel dieser Dokumentation ist es, Studierenden eine Anleitung zum
Erstellen wissenschatlicher Dokumente an die Hand zu geben. Dabei wird
auf Besonderheiten der zur Verf�gung gestellen \LaTeX -Klasse ebenso
eingegangen wie auf einige prinzipelle Formalismen wissenschatlicher
Arbeiten. Es wird aufgezeigt, wie Bilder oder Tabellen korrekt
erstellt und referenziert werden und wie ein Literaturverzeichnis
erstellt wird. Dieses Dokument selbst dient als Beispiel und kann als
Vorlage und Ausgangspunkt f�r eigene Arbeiten verwendet werden.

Es sei angemerkt, dass dies weder eine Einf�hrung in \LaTeX\ ist, noch wird
eine Anleitung zur Installation von \LaTeX -Distributionen gegeben.
Dazu sei deshalb auf die einschl�gige Literatur (zum Beispiel
\cite{Lamport95}, \cite{Goossens96}) oder Webseiten (\cite{WWWDante},
\cite{WWWMikTex}) verwiesen.

Weiterhin sei bemerkt, dass der Sprachstil dieser Arbeit �ber weite
Teile (nicht �berall) formaler gehalten ist als notwendig.
Der Grund hierf�r liegt in der Intention, dem Leser einen Eindruck zu
vermitteln, welcher Art der Stil einer wissenschaftlichen Arbeit sein
sollte. Man m�ge dem Autor deshalb den Mangel an Lockerheit und
humoristischen Ausschweifungen verzeihen.

% %%%%%%%%%%%%%%%%%%%%%%%%%%%%%%%%%%%%%%%%%%%%%%%%%%%%%%%%%%%%
% \section{Aufbau der Arbeit}
% \label{sec:einl:aufbau}
% Kapitel~\ref{chap:klasse} besch�ftigt sich zun�chst mit der von der AG
% Robotersysteme angebotenen \LaTeX -Klasse \latexklasse. Es wird
% erl�utert, welche Form das Hauptdokument haben muss und welche Makros
% zur Verf�gung gestellt werden.
% 
% Anschlie�end finden sich in Kapitel~\ref{chap:floats} Anleitungen zum
% Erstellen besonderer Dokumentelemente. Bilder und Tabellen werden
% ebenso behandelt wie Formeln oder Literaturverweise.
% 
% In Kapitel~\ref{chap:aufbau} werden einige kurze Hinweise zum
% allgemeinen Aufbau einer wissenschaftlichen Arbeit gegeben. Dies ist
% nur als Anregung zu verstehen, eine ausgiebige Einf�hrung hierzu w�rde
% den Rahmen dieser Dokumentation sprengen.


%%% Local Variables: 
%%% mode: latex
%%% TeX-master: "howto"
%%% End: 
