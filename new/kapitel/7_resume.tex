\chapter{Fazit und Ausblick}
\label{chap:resume}

Das Ziel dieser Arbeit war es eine iOS-Applikation zu entwickeln, welche eine Lokalisierung der aktuellen Position mittels iBeacons ermöglicht. Dabei sollte außerdem untersucht werden, in wie weit sich die Bluetooth LE-Technologie, auf welcher die iBeacons basieren, für die Indoor Positionierung eignen und welche Vorgehensweise bei der Lokaliserung die Geeignetste ist. Das Ziel, eine iOS-Applikation zu entwickeln, wurde dabei erfüllt. 

Bei der Untersuchung der Nutzbarkeit von Bluetooth bei dir Indoor Lokalisierung ist das Ergebnis nicht ganz zufriedenstellend. Die Funkttechnologie ist dabei sehr störanfällig und stark vom verwendeten Chipsatz und dem Design der Antenne abhängig. So waren zwischen den hier getesteten Beacons teilweise deutliche Unterschiede festzustellen. Auch die Störanfälligkeit durch sich bewegenden Menschen könnte eine Beeinflussung der Genauigkeit des Ergebnisses nach sich ziehen, da der Körper die Signalstärke deutlich abschwächt. Durch die Nutzung der Wahrscheinlichkeitsverteilung der Signalstärken bei der Positionierung war ein akzeptables Ergebnis möglich, welche jedoch nur für unbewegte Objekte eine zufriedenstellende Positionierung erreichte.

Das verwendete Gerät spielt bei der Positionierung ebenfalls eine große Rolle, da auch hier das Antennendesign und der verwendete Chipsatz einen großen Unterschied bei der empfangenen Signalstärke ausmachen. Dabei Unterscheiden sich selbst verschiedene Modelle des gleichen Herstellers deutlich in ihrer Empfangsstärke. 

Dadurch das diese ganzen Faktoren in die Positionierung mit einbezogen werden müssen, ist es schwierig ein einheitliches Verfahren für verschiedene Smartphonemodelle zu erstellen. Theoretisch müsste für jedes Gerät eine eigene Einmessung vorgenommen werden, wobei dies sowohl für die Smartphones als auch für die verwendeten Beacons gilt.
Eine Alternative wäre es die Abweichungen zwischen den einzelnen Geräten zu bestimmen. Aufgrund fehlender Testgeräte konnte in der Arbeit nicht bestimmt werden, in wie fern sich gleiche Modelle mit identischen Chipsätzen in der Signalqualität unterscheiden oder ob diese Werte vergleichbar sind.

Für die Zukunft ist Bluetooth für eine grobe Indoor Navigation und für die sogenannten Location-based Services eine vielversprechende Technologie. Die Bluetoothsender sind sehr günstig, einfach zu konfigurieren und in ihrer Anbringung äußerst flexibel. Durch Verbesserungen der Signalqualität der Bluetoothsender und der Antennen in den mobilen Geräten könnte die Positionierung zudem an Genauigkeit gewinnen. Dabei könnten zum Beispiel gerichtete Antennen zum Einsatz kommen, da deren Signal weniger verfälscht wird und so eine konstante Signalstärke beim Empfangsgerät ankommt. Die Lokalisierung mittels gerichteter Antennen wird momentan von Nokia untersucht, wobei deren System zwar auf Bluetooth 4.0 basiert, jedoch eigenständige Hardware benötigt.

Die Integrierung der iBeacons-Technologie in das Betriebssystem iOS wird die Bluetooth-Technolgie in der nächsten Zeit fördern und wahrscheinlich viele neue Einsatzmöglichkeiten für diese bieten. So wird die Nutzung von Location-based Services in der nächsten Zeit zunehmen, um zum Beispiel Gutscheine in Geschäften anzubieten oder automatisch Karten vom aktuellen Ort auf dem Gerät anzuzeigen.