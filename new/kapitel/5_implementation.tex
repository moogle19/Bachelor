\chapter{Umsetzung und Implementation}
\label{chap:implementation}

%%%%%%%%%%%%%%%%%%%%%%%%%%%%%%%%%%%%%%%%%%%%%%%%%%%%%%%%%%%%
\section{Ansatz zur Positionsbestimmung}
\label{sec:implementation:positioning}
%%%%%%%%%%%%%%%%%%%%%%%%%%%%%%%%%%%%%%%%%%%%%%%%%%%%%%%%%%%% 
Bei der Positionsbestimmung wurden zwei verschiedene Ansätze untersucht. Zum einen die Trilateration, welche eine Positionierung mittels Entfernungen zu verschiedenen Fixpunkten ermöglicht und zum Anderen die Positionierung mittels Fingerprinting, welches eine Datenbank mit sogenannten Fingerprints, also vorher aufgezeichneten Messwerten und damit verbundenen Positionsdaten, voraussetzt und darüber die aktuelle Position bestimmt.

Die Positionsbestimmung soll dabei in einem 2D-Raum erfolgen, da die Höhe vernachlässigt werden kann. In der realen Welt kann die Höhe ebenfalls vernachlässigt werden, da dort Stockwerke meist einen deutlichen Höhenunterschied aufweisen, sodass dieser über andere Faktoren eindeutig bestimmt werden kann.


%%%%%%%%%%%%%%%%%%%%%%%%%%%%%%%%%%%%%%%%%%%%%%%%%%%%%%%%%%%%
\section{Trilateration}
\label{sec:implementation:trilateration}
%%%%%%%%%%%%%%%%%%%%%%%%%%%%%%%%%%%%%%%%%%%%%%%%%%%%%%%%%%%%
Für die Trilateration im 2D-Raum werden mindestens drei Fixpunkte benötigt, um eine genaue Position zu erhalten.

\begin{figure}[htb!]
	\centering
	\includegraphics[scale=0.5]{trilateration}
	\caption{Funktionsprinzip der Trilateration}
	\label{trilateration}
\end{figure}


%%%%%%%%%%%%%%%%%%%%%%%%%%%%%%%%%%%%%%%%%%%%%%%%%%%%%%%%%%%%
\section{Fingerprinting}
\label{sec:implementation:fingerprinting}
%%%%%%%%%%%%%%%%%%%%%%%%%%%%%%%%%%%%%%%%%%%%%%%%%%%%%%%%%%%%