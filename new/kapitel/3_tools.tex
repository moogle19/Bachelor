\chapter{Werkzeuge}
\label{chap:tools}
In diesem Kapitel wird erläutert, welche Hilfsmittel bei der Erstellung des Bachelorarbeit genutzt werden.
Dabei wird näher auf die verwendete Hard- und Software eingegangen und wie sie für die Erstellung und die Tests genutzt wurde.


%%%%%%%%%%%%%%%%%%%%%%%%%%%%%%%%%%%%%%%%%%%%%%%%%%%%%%%%%%%%
\section{Xcode}
\label{sec:tools:xcode}
%%%%%%%%%%%%%%%%%%%%%%%%%%%%%%%%%%%%%%%%%%%%%%%%%%%%%%%%%%%%
Xcode ist eine integrierte Entwicklungsumgebung welche von Apple etwickelt wurde. Xcode ermöglicht es \emph{iOS} und \emph{OS X} Applikationen zu programmieren, zu testen und zu debuggen.
Standardmäßig werden dabei die Programmiersprachen \emph{Objective C}, \emph{C} und \emph{C++} unterstützt.

Xcode stellt viele Features für die Programmierung bereit, wie zum Beispiel \emph{code completion}, vorgefertigte \emph{Templates}, einen umfangreichen \emph{Debugger} und eine \emph{iOS-Simulator} für das Testen der Applikationen, ohne diese auf ein reales Gerät zu übertragen.

Bei Erstellung einer neuen Applikation kann man unter mehreren Templates wählen, welche jeweils verschiedene Funktionen mit sich bringen. In Abbildung \ref{xcode-templates} lassen sich die verschiedenen Auswahlmöglichkeiten erkennen.

\begin{figure}[htb!]
		\centering
	\includegraphics[scale=0.4]{xcode-templates}
	\caption{Auswahlbildschirm der verschiedenen Templates}
	\label{xcode-templates}
\end{figure}

Nach dem man das passende Template gewählt hat, werden die benötigten Dateien angelegt.
Dazu gehören beispielsweise die \emph{AppDelegate} und das \emph{Storyboard}.

Die AppDelegate-Klasse steuert applikationsweite Ereignisse, wie etwa das Aufrufen und Schließen der Applikation. Außerdem wird durch die AppDelegate der aktuelle Zustand der Applikation gespeichert und wiederhergestellt.

Das Storyboard ist eine grafische Oberfläche für die Erstellung des User Interfaces. Es ermöglicht verschiedene Elemente wie zum Beispiel Views, Textfelder, Buttons oder Tabellen einzufügen und diese zu verbinden. Wie in Abbildung \ref{xcode-storyboard} zu erkennen, besteht das Storyboard aus mehreren View Controllern, die jeweils eine gezeigte Szene auf dem Gerät repräsentieren. Die einzelnen View Controller sind mit so gennanten \emph{Segue's} verbunden, welche sich durch bestimmte Aktionen, wie zum Beispiel den Druck auf einen Button, auslösen lassen. Diese Segue's definieren, wie die Übergänge zwischen den ViewControllern ablaufen.

\begin{figure}[htb!]
	\centering
	\includegraphics[scale=0.25]{xcode-storyboard}
	\caption{Beispiel eines Storyboards für iPhones}
	\label{xcode-storyboard}
\end{figure}

Das Storyboard bietet außerdem noch die Funktion des \emph{Auto Layout}. Dabei werden sogenannte \emph{Constraints} genutzt, welche die Positionsbeziehungen zwischen den einzelnen Elementen festlegen. Diese Constraints erzeugen so ein dynamisches Interface, welches sich an das verwendete Gerät anpasst und so ein passendes User Interface, unabhängig der Bildschirmgröße oder der aktuellen Orientierung des Bildschirms, darstellt.
In Abbildung \ref{xcode-storyboard-constraints} lassen sich die Constraints, also die Abstands und Ausrichtungsbeziehungen zwischen den einzelnen Objekten, gut erkennen.

\begin{figure}[htb!]
		\centering
	\includegraphics[scale=0.5]{xcode-storyboard-constraints}
	\caption{View Controller mit Constraints}
	\label{xcode-storyboard-constraints}
\end{figure}

Diese Constraints beschreiben dabei die Abstände und Ausrichtungen der einzelnen Elemente zu dem umschließenden ViewController oder den anderen Elementen.

%%%%%%%%%%%%%%%%%%%%%%%%%%%%%%%%%%%%%%%%%%%%%%%%%%%%%%%%%%%%
\section{Objective-C}
\label{sec:tools:objectivec}
%%%%%%%%%%%%%%%%%%%%%%%%%%%%%%%%%%%%%%%%%%%%%%%%%%%%%%%%%%%%
Objective-C ist eine Programmiersprache, welche in den 80er Jahren entwickelt worden ist. Sie ist eine strikte Obermenge von C und erweitert diese um objektorientierte Konzepte. Objective-C ist die Hauptsprache für die Programmierung von Cocoa-Applikationen, wie sie unter iOS und OS X genutzt werden.



%%%%%%%%%%%%%%%%%%%%%%%%%%%%%%%%%%%%%%%%%%%%%%%%%%%%%%%%%%%%
\section{Versionsverwaltung mit Git}
\label{sec:tools:git}
%%%%%%%%%%%%%%%%%%%%%%%%%%%%%%%%%%%%%%%%%%%%%%%%%%%%%%%%%%%%
Xcode bietet für die Versionsverwaltung eine integrierte Git-Unterstützung, welche einfach und schnell zu bedienen ist.
Dabei werden die Differenzen innerhalb des Verzeichnisses bei einem Commit grafisch dargestellt und auch die Unterschiede innerhalb der Datei werden angezeigt.

\begin{figure}[htb!]
		  \centering
	\includegraphics[scale=0.25]{xcode-git-diff}
	\caption{Xcode Versionsverwaltung mit Diff-Anzeige bei einem Commit}
	\label{xcode-git-diff}
\end{figure}

Des Weiteren lassen sich die Änderungen auf Knopfdruck auf einen Server \emph{pushen} und vom Server \emph{pullen}.

Ausserdem lassen sich sehr einfach neue Branches erstellen und ein Mergen der Branches ist auch auf Knopfdruck möglich.


%%%%%%%%%%%%%%%%%%%%%%%%%%%%%%%%%%%%%%%%%%%%%%%%%%%%%%%%%%%%
\section{iOS Developer Program}
\label{sec:iosdevprogram}
%%%%%%%%%%%%%%%%%%%%%%%%%%%%%%%%%%%%%%%%%%%%%%%%%%%%%%%%%%%%
Um eine programmierte Anwendung letztendlich auf einem iOS-Gerät auszuführen, ist die Mitgliedschaft im iOS Developer Program notwendig.
Diese erlaubt das Testen der Anwendung auf dem Gerät, die Veröffentlichung im AppStore und gewährt Zugriff auf das iOS Beta-Programm, um Anwendungen für neue Versionen des Betriebssystems zu optimieren.
Die Mitgliedschaft in diesem \emph{iOS Developer Program} kostet jährlich 99 Dollar. 
Im Rahmen meine Bachelorarbeit wurde mir der Zugang zu diesem Programm von der Universität zur Verfügung gestellt.



%%%%%%%%%%%%%%%%%%%%%%%%%%%%%%%%%%%%%%%%%%%%%%%%%%%%%%%%%%%%
\section{iPhone}
\label{sec:tools:iphone}
%%%%%%%%%%%%%%%%%%%%%%%%%%%%%%%%%%%%%%%%%%%%%%%%%%%%%%%%%%%%
Für die Entwicklung und das Testen der Applikation wurde ein iPhone 5 und ein iPhone 4s verwendet. 
Hauptsächlich wurde das iPhone 5 genutzt, wobei das iPhone 4s eher als Vergleichsgerät diente, um zum Beispiel Messungen zu überprüfen.

Dabei geht es vor allem darum, zu überprüfen inwieweit die Ergebnisse der Messungen übertragbar sind beziehungsweise wie sie sich zwischen den einzelnen Modellen unterscheiden, da diese verschiedene Hardware einsetzen. So setzt das iPhone 4s auf den Broadcom BCM4330-Chipsatz, welches ein Wireless LAN-Chip mit integriertem Bluetooth 4.0 ist. Das iPhone 5 dagegen setzt auf den BCM4334, ebenfalls von Broadcom. 
Aber auch der Aufbau der Antennen und das Material der iPhones unterscheidet sich zwischen diesen beiden Generationen deutlich. 
So ist das iPhone 4s mit einer Glas-Rückseite ausgestattet, wohingegen das iPhone 5 einen Rückseite aus Aluminium besitzt.

Daher ist es wichtig zu vergleichen inwieweit die Änderungen die Empfangsqualität beeinflussen und zu bestimmen, ob eine Übertragung der Ergebnisse möglich ist oder ob jedes Gerät individuell behandelt werden muss.

\begin{figure}[htb!]
		\centering
	\includegraphics[scale=0.1]{iphones}
	\caption{Die für die Messungen und Tests genutzten iPhones}
	\label{iphones}
\end{figure}



