\chapter{Technologien}
\label{chap:technologies}

%%%%%%%%%%%%%%%%%%%%%%%%%%%%%%%%%%%%%%%%%%%%%%%%%%%%%%%%%%%%
\section{Bluetooth 4.0}
\label{sec:technologies:bluetooth4}
%%%%%%%%%%%%%%%%%%%%%%%%%%%%%%%%%%%%%%%%%%%%%%%%%%%%%%%%%%%%

Die Bluetooth-Version 4.0, oder auch Bluetooth Smart genannt, wurde 2009 final spezifiziert und wird seit Ende 2010 in Endgeräten eingesetzt.
Dieser Standard beinhaltet neben dem klassischen Bluetooth, eine neue Version, mit dem Namen Bluetooth Low Energy, welche, wie der Name schon andeutet, einen sehr viel geringeren Stromverbrauch vorweißt. Dabei ist der Stromverbrauch zwischen zwei und 100 mal geringer als beim klassischen Bluetooth.


%%%%%%%%%%%%%%%%%%%%%%%%%%%%%%%%%%%%%%%%%%%%%%%%%%%%%%%%%%%%
\section{Bluetooth Low Energy}
\label{sec:technologies:bluetoothLE}
%%%%%%%%%%%%%%%%%%%%%%%%%%%%%%%%%%%%%%%%%%%%%%%%%%%%%%%%%%%%

Bluetooth Low Energy wurde Anfangs von Nokia unter dem Namen ''Wibree'' entwickelt. Die Zielsetzung dabei war es eine Technologie zu entwickeln, mit der sich Computer und Mobilgeräte schnell und einfach mit Peripherie-Geräten verbinden lassen. Das Hauptaugenmerk galt dabei dem geringen Stromverbrauch, kompakter Bauweise und den Kosten für die benötigte Hardware.
Im Jahr 2007 wurden diese Spezifikationen dann in den, sich in der Entwicklung befindenden, Bluetooth-Standard 4.0 aufgenommen und daraufhin in Bluetooth Low Energy, oder kurz BLE umbenannt.

Bluetooth Low Energy arbeitet wie das klassische Bluetooth im 2,4 GHz Band, bringt aber in der Funktionsweise einige Unterschiede mit sich.

So wurde, im Vergleich zum klassischem Bluetooth, die Datenrate von bis zu 3 Mbit/s auf maximal 1 Mbit/s reduziert. Dies führt dazu, dass BLE zum Beispiel nicht für Headsets genutzt werden kann, da die zur Verffügung stehende Übertragungsrate nicht für die Audioübertragung ausreicht.

Die Vorteile die BLE mit sich bringt, liegen vor allem in der niedrigen Latenz, welche von 100ms auf bis zu unter 3ms reduziert wurde, und, wie bereits erwähnt, der Energieverbrauch drastisch gesenkt wurde.


%%%%%%%%%%%%%%%%%%%%%%%%%%%%%%%%%%%%%%%%%%%%%%%%%%%%%%%%%%%%
\subsection{iBeacons}
\label{sec:technologies:bluetoothLE:ibeacons}
%%%%%%%%%%%%%%%%%%%%%%%%%%%%%%%%%%%%%%%%%%%%%%%%%%%%%%%%%%%%


%%%%%%%%%%%%%%%%%%%%%%%%%%%%%%%%%%%%%%%%%%%%%%%%%%%%%%%%%%%%
\section{iOS und Xcode}
\label{sec:technologies:iosandxcode}
%%%%%%%%%%%%%%%%%%%%%%%%%%%%%%%%%%%%%%%%%%%%%%%%%%%%%%%%%%%%


%%%%%%%%%%%%%%%%%%%%%%%%%%%%%%%%%%%%%%%%%%%%%%%%%%%%%%%%%%%%
\section{CoreLocation-Framework}
\label{sec:technologies:corelocation}
%%%%%%%%%%%%%%%%%%%%%%%%%%%%%%%%%%%%%%%%%%%%%%%%%%%%%%%%%%%%


%%%%%%%%%%%%%%%%%%%%%%%%%%%%%%%%%%%%%%%%%%%%%%%%%%%%%%%%%%%%
\subsection{iBeacons-API}
\label{sec:technologies:iosandxcode:ibeaconsapi}
%%%%%%%%%%%%%%%%%%%%%%%%%%%%%%%%%%%%%%%%%%%%%%%%%%%%%%%%%%%%

%%%%%%%%%%%%%%%%%%%%%%%%%%%%%%%%%%%%%%%%%%%%%%%%%%%%%%%%%%%%
\subsection{Weitere API's}
\label{sec:technologies:iosandxcode:otherapis}
%%%%%%%%%%%%%%%%%%%%%%%%%%%%%%%%%%%%%%%%%%%%%%%%%%%%%%%%%%%%


%%%%%%%%%%%%%%%%%%%%%%%%%%%%%%%%%%%%%%%%%%%%%%%%%%%%%%%%%%%%
\section{CoreData-Framework}
\label{sec:technologies:coredata}
%%%%%%%%%%%%%%%%%%%%%%%%%%%%%%%%%%%%%%%%%%%%%%%%%%%%%%%%%%%%