\chapter{Technologien}
\label{chap:technologies}

%%%%%%%%%%%%%%%%%%%%%%%%%%%%%%%%%%%%%%%%%%%%%%%%%%%%%%%%%%%%
\section{Bluetooth 4.0}
\label{sec:technologies:bluetooth4}
%%%%%%%%%%%%%%%%%%%%%%%%%%%%%%%%%%%%%%%%%%%%%%%%%%%%%%%%%%%%

Die Bluetooth-Version 4.0, oder auch Bluetooth Smart genannt, wurde 2009 final spezifiziert und wird seit Ende 2010 in Endgeräten eingesetzt.
Dieser Standard beinhaltet neben dem klassischen Bluetooth, eine neue Version, mit dem Namen Bluetooth Low Energy, welche, wie der Name schon andeutet, einen sehr viel geringeren Stromverbrauch vorweißt. Dabei ist der Stromverbrauch zwischen zwei und 100 mal geringer als beim klassischen Bluetooth.


%%%%%%%%%%%%%%%%%%%%%%%%%%%%%%%%%%%%%%%%%%%%%%%%%%%%%%%%%%%%
\section{Bluetooth Low Energy}
\label{sec:technologies:bluetoothLE}
%%%%%%%%%%%%%%%%%%%%%%%%%%%%%%%%%%%%%%%%%%%%%%%%%%%%%%%%%%%%

Bluetooth Low Energy wurde Anfangs von Nokia unter dem Namen ''Wibree'' entwickelt. Die Zielsetzung dabei war es eine Technologie zu entwickeln, mit der sich Computer und Mobilgeräte schnell und einfach mit Peripherie-Geräten verbinden lassen. Das Hauptaugenmerk galt dabei dem geringen Stromverbrauch, kompakter Bauweise und den Kosten für die benötigte Hardware.
Im Jahr 2007 wurden diese Spezifikationen dann in den, sich in der Entwicklung befindenden, Bluetooth-Standard 4.0 aufgenommen und daraufhin in Bluetooth Low Energy, oder kurz BLE umbenannt.

Bluetooth Low Energy arbeitet wie das klassische Bluetooth im 2,4 GHz Band, bringt aber in der Funktionsweise einige Unterschiede mit sich.

So wurde, im Vergleich zum klassischem Bluetooth, die Datenrate von bis zu 3 Mbit/s auf maximal 1 Mbit/s reduziert. Dies führt dazu, dass BLE zum Beispiel nicht für Headsets genutzt werden kann, da die zur Verffügung stehende Übertragungsrate nicht für die Audioübertragung ausreicht.

Die Vorteile die BLE mit sich bringt, liegen vor allem in der niedrigen Latenz, welche von 100ms auf bis zu unter 3ms reduziert wurde, und, wie bereits erwähnt, der Energieverbrauch drastisch gesenkt wurde.



Bluetooth Low Energy bietet darüber hinaus eine Vielzahl sogennanter GATT-Profile (Generic Attribute Profile). Die bereitgestellten GATT-Profile sind Richtlinen für die Bluetooth-Funktionialität, sprich, welche Daten übertragen werden und in welcher Form. Dies erlaubt eine einfache und schnell Interoperabilität zwischen verschiedenen Geräten. Ein Beispiel für ein GATT-Profil wäre zum Beispiel das ''Heart Rate Profile'', welches besipielsweise die Verbindung und Kommunikation eines Pulsmessgurtes mit einem Endgerät beschreibt. So wird sichergestellt, dass dieser Gurt mit jedem Endgerät auf die selbe Weise funktioniert.




%%%%%%%%%%%%%%%%%%%%%%%%%%%%%%%%%%%%%%%%%%%%%%%%%%%%%%%%%%%%
\subsection{iBeacons}
\label{sec:technologies:bluetoothLE:ibeacons}
%%%%%%%%%%%%%%%%%%%%%%%%%%%%%%%%%%%%%%%%%%%%%%%%%%%%%%%%%%%%
Die iBeacons-Technologie wurde am 10.Juni 2014 von Apple auf der Worldwide Developers Conference vorgestellt. 
Diese basiert auf Bluetooth Low Energy und arbeitet mit einem von Apple entwickelten GATT-Profil.

Beacon bedeutet übersetzt ''Leuchtfeuer'' und die Funktionsweise der Beacons ist dem sehr ähnlich.
Einmal in Betrieb genommen, sendet das Beacon kontinuierlich ein Signal, in welchem sich Daten zur Identifizierung des Beacons befinden.

Neben den Identifikationsdaten kann das Emfangsgerät noch weitere Größen bestimmen. Es ist so zum Beispiel möglich die ungefähre Entfernung einzuschätzen. 
In der iBeacons-API sind dafür vier verschiedene Zustände definiert: \textit{Far}, \textit{Near}, \textit{Immediate} und \textit{Unknown}. Diese Werte erlauben eine grobe Entfernungseinschätzung und für eine genauere Bestimmung lässt sich noch eine weitere Kenngröße bestimmen, der \textit{Accuracy}-Wert. Dabei handelt es sich um eine ungefähre Entfernungsangabe in Metern, welche jedoch ausdrücklich nur zur Differenzierung zwischen zwei Beacons genutzt werden soll und keinesfalls eine genaue Entfernung angibt.

\begin{tabular}{p{2cm}p{5cm}p{5cm}p{4cm}}
	\\
	Daten & Format & Beschreibung & Beispiel \\ \\
	UUID & 16-stellige Hexadezimalzahl & Identifizierung & 3F4 \\
	Major & Integerzahl & Identifizierung eine Region & 12 \\
	Minor & Integerzahl & Identifizierung eines einzelnen Beacons & 132 \\
	Proximity & Drei Entfernungsstufen & Ungefähre Entfernung & Far, Near, Immediate und Unknown \\
	Accuracy & Wert in Meter & Bestimmung der ungefähren Entfernung & 1.243 m \\
	RSSI & Signalstärke in dBm & Signalstärke des emfangenen Signals & -42 dBm \\
\end{tabular}


Die von dem Beacon gesendeten Daten lassen sich mit jedem BLE-kompatiblem Gerät empfangen.



Die großen Vorteile der iBeacons sind zum einen ihr kleiner Formfaktor, welcher es erlaubt die Beacons an fast jedem beliebigem Ort anzubringen, als auch ihr geringer Stromverbrauch, der es möglich macht, die Beacons bis zu mehreren Jahren mit einer Knopfzellenbatterie zu betreiben. Der Aufbau eines solchen Beacons lässt sich in Abbildung \ref{estimote-beacon} sehr gut erkennen. Den Großteil des Beacons nimmt dabei die Batterie ein. 


%%%%%%%%%%%%%%%%%%%%%%%%%%%%%%%%%%%%%%%%%%%%%%%%%%%%%%%%%%%%
% figure of estimote beacon
%%%%%%%%%%%%%%%%%%%%%%%%%%%%%%%%%%%%%%%%%%%%%%%%%%%%%%%%%%%%
\begin{figure}[h!]
	\centering
	\begin{minipage}[t]{5cm}
		\includegraphics[scale=0.15]{pictures/estimote-beacon-outside}
		\caption{Außenhülle}
		\label{estimote-outside}
	\end{minipage}
	\hspace{2cm}
	\begin{minipage}[t]{5cm}
			\includegraphics[scale=0.2]{pictures/estimote-beacon-inside}
			\caption{Chipsatz mit Bluetooth-Modul}
			\label{estimote-inside}
	\end{minipage}
		\caption{Ein iBeacon der Firma ''estimote''}
		\label{estimote-beacon}
\end{figure}


Unter genauerer Betrachtung des Chipsatzes in Abbildung \ref{estimote-beacon-inside-annotations}, erkennt man, dass er im Grunde aus zwei Teilen besteht.
Dem Bluetooth-Chipsatz, welcher an sich ist nur wenige Zentimeter groß und der Antenne, welche im vorderen Bereich der Platine eingearbeitet ist und die über welche letztendlich die Daten gesendet werden.

\begin{figure}[h!]
	\centering
	\includegraphics[scale=0.25]{estimote-beacon-inside-annotation}
	\caption{Aufbau des estimote-Beacons}
	\label{estimote-beacon-inside-annotations}
\end{figure}



%%%%%%%%%%%%%%%%%%%%%%%%%%%%%%%%%%%%%%%%%%%%%%%%%%%%%%%%%%%%
\section{iOS, OS X und Xcode}
\label{sec:technologies:iosandxcode}
%%%%%%%%%%%%%%%%%%%%%%%%%%%%%%%%%%%%%%%%%%%%%%%%%%%%%%%%%%%%
Für die Entwicklung der Applikation zur Indoor Positionierung war eine der Vorgaben, dass diese für iOS programmiert werden soll.
Daher waren drei Dinge zwingend notwendig: ein Mac, Xcode und ein iOS-Gerät.

Für die Entwicklung setzte ich deshalb auf ein Macbook Pro mit installiertem Xcode und als iOS-Gerät setzte ich ein iPhone 5 ein.
Als minimale iOS-Version musste iOS 7 verwendet werden, da die iBeacon-Features des CoreLocation-Frameworks (mehr dazu im Kapitel \ref{sec:technologies:corelocation}) erst ab dieser Version zur Verfügung stehen.


%%%%%%%%%%%%%%%%%%%%%%%%%%%%%%%%%%%%%%%%%%%%%%%%%%%%%%%%%%%%
\section{CoreLocation-Framework}
\label{sec:technologies:corelocation}
%%%%%%%%%%%%%%%%%%%%%%%%%%%%%%%%%%%%%%%%%%%%%%%%%%%%%%%%%%%%
Das Core Location-Framework erlaubt es aktuelle Positions- und Richtungsinformationen eines Gerätes zu bestimmen.
Die Positionsbestimmung lässt sich dabei über verschiedene Werte und Sensoren bestimmen und auch der Grad der Genauigkeit ist variabel.
Auch die Aktualiserungsrate der Position lässt sich festlegen, wobei eine höhere Aktualisierungsrate auch gleichbedeutend mit einem höherem Akkuverbrauch ist.

Bei der Genauerigkeit gibt es dabei verschiedene Konstanten, die die gewollte Genauigkeit bestimmen. 


    \begin{table}[htb!]
      \centering
      \begin{tabular}{l p{6cm}}
        Konstante & Erwartete Genauigkeit \\ \\
		\emph{kCLLocationAccuracyThreeKilometers} & Genauigkeit auf 3 Kilometer \\
		\emph{kCLLocationAccuracyKilometer} & Genauigkeit auf 1 Kilometer \\
		\emph{kCLLocationAccuracyHundredMeters} & Genauigkeit auf 100 Meter \\
		\emph{kCLLocationAccuracyNearestTenMeters} & Genauigkeit auf 10 Meter \\
		\emph{kCLLocationAccuracyBest} & Höchstmögliche Genauigkeit \\
		\emph{kCLLocationAccuracyBestForNavigation} & Höchstmögliche Genauigkeit und weitere Sensordaten für die Navigation
      \end{tabular}
      \caption{Mögliche Optionen der Positionsgenauigkeit}
      \label{tbl:positionaccuracy}
    \end{table}
	
Diese Genauigkeiten beziehen sich hauptsächlich auf die Positionierung mittels GPS und sind daher für die Indoor Positionierung nur bedingt geeignet.


Eine weitere Funktion des Core Location-Frameworks ist die Bestimmung der Himmelsrichtungen. Durch den eingebauten Kompass in den neueren iOS-Geräten ist es möglich, die aktuelle Ausrichtung des Gerätes zu bestimmen. Dies ist im Bezug auf die Indoor Navigation hilfreich, da diese Informationen in die Positionsbestimmung einbezogen werden können.
Des Weiteren erlaubt diese Funktion eine dynamische Ausrichtung der Karte, abhängig davon in welche Richtung man momentan schaut.


Die für uns zentrale Funktion dieses Frameworks ist die Erkennung von iBeacons und die Funktionen zur Verarbeitung der gesendeten Daten.
Damit können Beacons anhand ihres UUID erkannt werden und einer Region zugeordnet werden. Die genaue Funktionsweise wird in Kapitel \ref{sec:technologies:corelocation:ibeaconsapi}.

Die Funktionen zum Positionsupdate und zur Erkennung der Beacons werden dabei im \emph{LocationManager} verwaltet.
In der \emph{LocationManagerDelegate} lassen sich dabei die Aktionen bestimmen, welche bei verschiedenen Events ausgeführt werden.

In Listing \ref{lst:locationmanager_objc} wird die Initialisierung eines LocationManager gezeigt, welcher eine Genauigkeit von einem Kilometer haben soll und bei Positionsänderungen von mehr als 500 Metern aktualisiert wird.

  \begin{listing}[htb!]
    \insertminted{objc}{code_examples/locationManager.m}
    \caption{Beispielinitialisierung für einen LocationManager.}
    \label{lst:locationmanager_objc}
  \end{listing}

%%%%%%%%%%%%%%%%%%%%%%%%%%%%%%%%%%%%%%%%%%%%%%%%%%%%%%%%%%%%
\subsection{iBeacons-API}
\label{sec:technologies:corelocation:ibeaconsapi}
%%%%%%%%%%%%%%%%%%%%%%%%%%%%%%%%%%%%%%%%%%%%%%%%%%%%%%%%%%%%
Seit der iOS Version 7 wurde das Core Location Framework um die Beacon-Funktionen erweitert. 
Dazu wurden zwei neue Klassen geschaffen. Einmal die \emph{CLBeacon}-Klasse, welche ein iBeacon repräsentiert und alle zur verfügungsteheneden Informationen enthält und zum anderen die \emph{CLBeaconRegion}-Klasse, welche eine Region mit mehreren Beacons, abhängig von ihrem UUID, beschreibt.

Die \emph{CLBeacon}-Klasse besteht dabei lediglich aus Propertys mit den gegenbenden Beacon-Informatinen, wie \emph{UUID}, \emph{major}, \emph{minor}, \emph{accuracy}, \emph{proximity} und \emph{rssi}.

Die \emph{CLBeaconRegion}-Klasse ist etwas umfangreicher und bestimmt letztendlich, nach welchen Beacons gesucht werden soll.
Dabei ist es möglich die Region in verschiedene Genaugikeitsstufen einzuteilen.


\emph{initWithProximityUUID:identifier:}\begin{quote}
	Die Region ist nur abhängig von dem UUID und dem Identifier der Beacons, das heißt es werden alle Beacons mit dem gegebenen UUID gesucht.
\end{quote}
\emph{initWithProximityUUID:major:identifier:}\begin{quote}
	Die Region ist abhängig von dem UUID, dem Identifier und dem Major-Wert der Beacons. Es werden nur Beacons eines bestimmten Major-Wertes gesucht.
\end{quote}
\emph{initWithProximityUUID:major:minor:identifier:}\begin{quote}
	Die Region ist abhängig von dem UUID, dem Identifier, dem Major-Wert und dem Minor-Wert der Beacons. Es werden nur Beacons mit passendem Major und Minor-Wert gesucht. In diesem Fall ist bei mehreren erkannten Beacons keine Unterscheidung mehr möglich.
\end{quote}



%%%%%%%%%%%%%%%%%%%%%%%%%%%%%%%%%%%%%%%%%%%%%%%%%%%%%%%%%%%%
\section{MapBox-Framework}
\label{sec:sec:technologies:mapbox}
%%%%%%%%%%%%%%%%%%%%%%%%%%%%%%%%%%%%%%%%%%%%%%%%%%%%%%%%%%%%

%%%%%%%%%%%%%%%%%%%%%%%%%%%%%%%%%%%%%%%%%%%%%%%%%%%%%%%%%%%%
\subsection{Weitere API's}
\label{sec:technologies:iosandxcode:otherapis}
%%%%%%%%%%%%%%%%%%%%%%%%%%%%%%%%%%%%%%%%%%%%%%%%%%%%%%%%%%%%


%%%%%%%%%%%%%%%%%%%%%%%%%%%%%%%%%%%%%%%%%%%%%%%%%%%%%%%%%%%%
\section{CoreData-Framework}
\label{sec:technologies:coredata}
%%%%%%%%%%%%%%%%%%%%%%%%%%%%%%%%%%%%%%%%%%%%%%%%%%%%%%%%%%%%